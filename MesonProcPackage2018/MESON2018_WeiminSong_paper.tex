%%%%%%%%%%%%%%%%%%%%%%% file MESON2018_YourName_paper.tex %%%%%%%%
%
% This is a template file for Web of Conferences Journal
%
% Copy it to a new file with a new name and use it as the basis
% for your article
%
%%%%%%%%%%%%%%%%%%%%%%%%%% EDP Science %%%%%%%%%%%%%%%%%%%%%%%%%%%%
%
%%%\documentclass[option]{webofc}
%%% "twocolumn" for typesetting an article in two columns format (default one column)
%
\documentclass[epj]{webofc}
\usepackage[varg]{txfonts}   % Web of Conferences font
%
% Put here some packages required or/and some personnal commands
%
\woctitle{MESON2018 - the 15$^\textrm{th}$ International Workshop on Meson Physics}
%
\begin{document}
%
\selectlanguage{english}
\title{Recent results of quarkonium and heavy flavour at ATLAS}

% insert email only for speaker/presenter
\author{Weimin~Song\inst{1}\fnsep\thanks{\email{wesong@cern.ch}}
% comment out the next line if not needed
       \\for the ATLAS Collaboration
}

\institute{Particle Physics Department, Rutherford Appleton Laboratory, Didcot, UK
          }

\abstract{%Do not break line here!
Heavy quark spectroscopy and exotic states are studied with the ATLAS detector, mainly
thorough final states containing muon pairs from $J/\psi$ decays from both proton-proton,
proton-lead and lead-lead collisions. This proceedings will summarise recent results from ATLAS,
including production of quarkonium and heavy flavour, searches for exotic states and
measurements of decay properties in open beauty production.
}
%
\maketitle
%
\section{Introduction}
\label{intro}

Hardons have been used to test the particle physics Standard Model (SM) for a long time.
%In the current days, they are still crutial for particle physics, for examle by measuring the decay rate
%of $B_s \to \mu^{+}\mu^{-}$, the SM could be checked in a very high precision.
The heavy flavour hadrons are massly produced in the proton proton collisions at Large Hadron Collider.
With ATLAS detector, the production and decay of the heavy flavour hadrons could be studied. 
When comparing with the other detectors which are optimised for hadron physics study, such as Belle (II), BESIII, and
LHCb, the advantage of ATLAS is that the number of recorded hardons is larger, while the disadvantage is the particle identification 
system is not designed for the seperation between different hadrons. 
Normally, the $J/\psi \to \mu^{+} \mu^{-}$ is required in the final state, in order to reduce the background, which is much more
higher in the hadron collider when comparing with electron-positron collider.\\ 

In this proceddings, four recent publications from ATLAS are discussed, they are:
\begin{itemize}
  \item Search for a Structure in the $B^0_s \pi^\pm$ Invariant Mass Spectrum with the ATLAS Experiment~\cite{x5568}
  \item Measurement of $b$-hadron pair production with the ATLAS detector in proton-proton collisions at $\sqrt{s}=8$ TeV~\cite{bpair}
  \item Measurement of quarkonium production in proton–lead and proton–proton collisions at $5.02~\mathrm {TeV}$ with the ATLAS detector~\cite{quark_pro}
  \item Angular analysis of $B^0_d \rightarrow K^{*}\mu^+\mu^-$ decays in $pp$ collisions at $\sqrt{s}= 8$ TeV with the ATLAS detector~\cite{kmumu}
\end{itemize}

\section{Is X(5568) observed in the ATLAS dataset?}

In 2016, with 10.4~$fb^(-1)$ proton anti-proton collision data at center of mass energy of 1.96 TeV, the D0 collaboration made a claim about
the discovery of X(5568) in the $B^0_s \pi^\pm$ final states. 

\label{5568}

\section{Summary}
\label{sum}

\begin{acknowledgement}

\end{acknowledgement}
%
% BibTeX or Biber users please use (the style is already called in the class, ensure that the "woc.bst" style is in your local directory)
% \bibliography{name or your bibliography database}
%
% Non-BibTeX users please use
%
\begin{thebibliography}{00}
%
% and use \bibitem to create references.
%
\bibitem{x5568}
  M.~Aaboud {\it et al.} [ATLAS Collaboration],
  %``Search for a Structure in the $B^0_s \pi^\pm$ Invariant Mass Spectrum with the ATLAS Experiment,''
  Phys.\ Rev.\ Lett.\  {\bf 120}, no. 20, 202007 (2018)
\bibitem{bpair}
  M.~Aaboud {\it et al.} [ATLAS Collaboration],
  %``Measurement of $b$-hadron pair production with the ATLAS detector in proton-proton collisions at $\sqrt{s}=8$ TeV,''
  JHEP {\bf 1711}, 062 (2017)
\bibitem{quark_pro}
  M.~Aaboud {\it et al.} [ATLAS Collaboration],
  %``Measurement of quarkonium production in proton–lead and proton–proton collisions at $5.02~\mathrm {TeV}$ with the ATLAS detector,''
  Eur.\ Phys.\ J.\ C {\bf 78}, no. 3, 171 (2018)
\bibitem{kmumu}
  M.~Aaboud {\it et al.} [ATLAS Collaboration],
  %``Angular analysis of $B^0_d \rightarrow K^{*}\mu^+\mu^-$ decays in $pp$ collisions at $\sqrt{s}= 8$ TeV with the ATLAS detector,''
  arXiv:1805.04000 [hep-ex].
\end{thebibliography}

\end{document}

% end of file template.tex

<div id='footer'><table width='100%'><tr><td class='right'><a href='http://fusioninventory.org/'><span class='copyright'>FusionInventory 9.1+1.0 | copyleft <img src='/glpi/plugins/fusioninventory/pics/copyleft.png'/>  2010-2016 by FusionInventory Team</span></a></td></tr></table></div>
